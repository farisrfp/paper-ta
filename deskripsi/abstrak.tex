% Mengubah keterangan `Abstract` ke bahasa indonesia.
% Hapus bagian ini untuk mengembalikan ke format awal.
\renewcommand\abstractname{Abstrak}

\begin{abstract}

  Keseimbangan dan stabilitas menjadi aspek kunci dalam pengembangan robot humanoid, terutama ketika robot harus melakukan aktivitas yang melibatkan gerakan kaki dengan tumpuan hanya pada satu kaki. Penelitian ini bertujuan untuk mengembangkan sistem keseimbangan pada robot humanoid dengan menggunakan sensor \emph{load cell} pada telapak kaki. Robot humanoid yang digunakan dalam penelitian ini menggunakan robot VI-ROSE ITS dengan total derajat kebebasan 29 DoF. Sistem ini memonitor perubahan tekanan pada kedua telapak kaki robot dan menghitung titik pusat tekanan (\textit{Center of Pressure}) terhadap robot. Perhitungan pusat teknanan tersebut menggunakan empat buah \emph{load cell} pada masing-masing telapak kaki yang diproses oleh mikrokontroler terletak pada telapak kaki dan dikirimkan ke mikrokontroler utama secara nirkabel. Kesalahan pengukuran sensor berkisar antara 0-50 gram dengan rata-rata 14.8 gram di kaki kanan dan 14.6 gram di kaki kiri. Sensor dapat mendeteksi pusat tekanan pada sumbu X dengan nilai maksimum 1.44 saat mengangkat kaki kanan dan minimum -1.33 saat mengangkat kaki kiri. Robot mampu menyeimbangkan diri menggunakan kontrol PID pada 5 servo untuk mengatur posisi \textit{roll}, dengan nilai \(K_p\) = 0.10 dan \(K_d\) = 0.005, mencapai tingkat keberhasilan 100\% dalam menjaga keseimbangan saat mengangkat kaki dengan kemiringan 3 derajat.

\end{abstract}

% Hapus bagian ini untuk mengembalikan ke format awal.
\renewcommand\IEEEkeywordsname{Kata kunci}

\begin{IEEEkeywords}

  % Ubah kata-kata berikut sesuai dengan kata kunci dari penelitian.
  Robot Humanoid, Sensor \textit{Load Cell}, Pusat Tekanan

\end{IEEEkeywords}
