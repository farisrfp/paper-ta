% Mengubah keterangan `Abstract` ke bahasa indonesia.
% Hapus bagian ini untuk mengembalikan ke format awal.
\renewcommand\abstractname{Abstrak}

\begin{abstract}

  % Ubah paragraf berikut sesuai dengan abstrak dari penelitian.
  Keseimbangan dan stabilitas menjadi aspek kunci dalam pengembangan robot humanoid, terutama ketika robot harus melakukan aktivitas yang melibatkan gerakan kaki dengan tumpuan hanya pada satu kaki. Penelitian ini bertujuan untuk mengembangkan sistem keseimbangan pada robot humanoid dengan menggunakan sensor \emph{load cell} pada telapak kaki. Robot humanoid yang digunakan dalam penelitian ini menggunakan robot VI-ROSE ITS. Sistem keseimbangan yang dikembangkan akan memonitor perubahan beban pada kedua telapak kaki robot dan menghitung titik pusat tekanan (\textit{Center of Pressure}) terhadap robot. Sistem keseimbangan berbasis \emph{load cell} pada robot humanoid VI-ROSE ITS dengan menggunakan empat \emph{load cell} yang diproses oleh mikrokontroler ESP32-C3 dan dikirimkan ke mikrokontroler utama melalui protokol ESP-NOW. Data yang diterima digunakan untuk mendeteksi posisi pusat tekanan dan mengontrol gerakan robot agar dapat berdiri dan berjalan dengan stabil menggunakan kontrol PID. Pemilihan ESP32-C3 didasari oleh kemampuannya mengolah data dengan cepat dan ukurannya yang kecil, memungkinkan penempatan pada telapak kaki robot. Hasil penelitian menunjukkan bahwa sistem ini efektif menjaga keseimbangan robot pada bagian \textit{roll} selama fase \textit{single phase support}, dengan nilai koreksi pada servo yang mengatur posisi \textit{roll}. Pengujian kontrol PID menunjukkan bahwa nilai optimal parameter \(K_p\) adalah antara 0.10 hingga 0.20 untuk performa terbaik, sementara parameter \(K_i\) tidak terlalu signifikan, dan \(K_d\) optimal antara 0.005 hingga 0.010, dengan nilai yang terlalu tinggi menyebabkan penurunan performa sistem. 

\end{abstract}

% Mengubah keterangan `Index terms` ke bahasa indonesia.
% Hapus bagian ini untuk mengembalikan ke format awal.
\renewcommand\IEEEkeywordsname{Kata kunci}

\begin{IEEEkeywords}

  % Ubah kata-kata berikut sesuai dengan kata kunci dari penelitian.
  Robot Humanoid, Load Cell Sensor, Center of Pressure

\end{IEEEkeywords}
