% Mengubah keterangan `Abstract` ke bahasa indonesia.
% Hapus bagian ini untuk mengembalikan ke format awal.
% \renewcommand\abstractname{Abstrak}

\begin{abstract}

  % Ubah paragraf berikut sesuai dengan abstrak dari penelitian.
  Balance and stability are crucial aspects in the development of humanoid robots, especially when the robot needs to perform activities involving single-leg support. This research aims to develop a balance system for a humanoid robot using load cell sensors on the feet. The humanoid robot used in this research is the VI-ROSE ITS robot. The developed balance system monitors the load changes on both feet of the robot and calculates the Center of Pressure (CoP). The balance system based on load cell for the VI-ROSE ITS humanoid robot uses four load cells processed by the ESP32-C3 microcontroller and transmitted to the main microcontroller via the ESP-NOW protocol. The received data is used to detect the CoP position and control the robot's movements to stand and walk stably using PID control. The selection of the ESP32-C3 is based on its ability to process data quickly and its small size, allowing it to be placed on the robot's feet. The research results indicate that this system effectively maintains the robot's balance in the roll part during the single-phase support phase, with correction values applied to the servo controlling the roll position. PID control testing shows that the optimal parameter \(K_p\) is between 0.10 and 0.20 for the best performance, while the \(K_i\) parameter is not very significant, and the optimal \(K_d\) parameter is between 0.005 and 0.010, with higher values causing a decrease in system performance.


\end{abstract}

% Mengubah keterangan `Index terms` ke bahasa indonesia.
% Hapus bagian ini untuk mengembalikan ke format awal.
% \renewcommand\IEEEkeywordsname{Kata kunci}

\begin{IEEEkeywords}

  % Ubah kata-kata berikut sesuai dengan kata kunci dari penelitian.
  Humanoid Robot, Load Cell Sensor, Center of Pressure

\end{IEEEkeywords}
