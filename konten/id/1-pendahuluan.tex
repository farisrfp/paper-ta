% Ubah judul dan label berikut sesuai dengan yang diinginkan.
\section{Pendahuluan}
\label{sec:pendahuluan}

Dalam beberapa tahun terakhir, perkembangan pesat dalam robotika telah meningkatkan interaksi antara manusia dan robot, khususnya dalam konteks robot humanoid. Menjaga keseimbangan dan stabilitas adalah aspek kunci dalam perancangan robot humanoid \cite{chiang2020posture}. Hal ini menjadi sangat penting dalam perlombaan robot tari tradisional, di mana robot tidak diperbolehkan untuk jatuh untuk melewati arena dan harus mampu menampilkan tarian dengan gerakan yang kompleks. Manusia memiliki kemampuan untuk menyeimbangkan diri dengan cepat dan efisien, bahkan ketika berjalan di permukaan yang tidak rata. 

Penelitian oleh Ilyasaa (2023)\cite{rotama} membahas implementasi sensor load cell pada robot humanoid dengan 24 derajat kebebasan untuk mengontrol keseimbangan saat berjalan. Sensor ini mendeteksi tekanan pada kaki robot, digunakan sebagai input kontrol PID untuk mengatur parameter pitch langkah robot. Hasil menunjukkan robot dapat berdiri pada sudut miring 0°-25° dan mengontrol pitch untuk mengurangi kemungkinan jatuh, namun pada sudut 30° robot mengalami kesulitan dan cenderung jatuh. Kelebihan penelitian ini adalah penggunaan kontrol PID yang efektif, namun kurangnya analisis pengaruh sudut miring terhadap pengaturan robot saat berjalan.

Wulandari (2023)\cite{lanangjagad} membahas penggunaan sensor Load Cell, Kalman Filter, dan PID Controller untuk keseimbangan robot tari humanoid, yang harus stabil di berbagai kondisi medan kompetisi KRSTI. Sistem kontrol menggunakan sensor MPU6050 dan Load Cell untuk mendeteksi Center of Pressure (CoP) dan menjaga keseimbangan. Hasil menunjukkan robot berhasil mempertahankan keseimbangan dengan tingkat keberhasilan 87.5\% saat berdiri dan 89\% saat menari. Kelebihan penelitian ini adalah kombinasi sensor dan algoritma yang efektif, namun pengujian hanya dilakukan pada fase double support, belum mempertimbangkan gerakan pada fase single support kemudian desain \emph{hardware} yang masih menggunakan kabel sehingga kaki robot tidak dapat bergerak dengan bebas.

Robot yang akan diteliti terkadang masih jatuh saat melakukan tarian, terutama ketika berjalan atau mengangkat satu kaki. Pada robot tari humanoid yang biasa digunakan untuk Kontes Robot Seni Tari (KRSTI), penambahan aksesoris pada kostum semakin menantang kemampuan sistem untuk menjaga keseimbangan. Untuk mengatasi ini, diperlukan peningkatan respons sistem terhadap perubahan keseimbangan. Sensor \emph{load cell} dapat memberikan informasi mengenai pusat tekanan pada robot. Sensor \emph{load cell} dapat mendeteksi perubahan detail tekanan yang terjadi. Hal ini memungkinkan sistem untuk mengukur perubahan pusat keseimbangan dengan lebih tepat dan meresponsnya dengan cepat. Selain itu, untuk meningkatkan efisiensi pengiriman data dari sensor \emph{load cell} yang terletak pada telapak kaki, diusulkan penerapan teknologi nirkabel menggunakan sinyal radio berfrekuensi 2.4GHz.

Penelitian ini bertujuan untuk mengembangkan sistem tertanam pada telapak kaki robot tari humanoid dengan 29 derajat kebebasan yang mampu membaca pusat tekanan menggunakan sensor load cell dan menerapkan sistem kontrol dengan mengontrol beberapa servo untuk menyeimbangkan robot saat bergerak di bidang miring. Batasan masalah penelitian ini meliputi: penggunaan mikrokontroler ESP32, jumlah derajat kebebasan bagian kaki robot masing-masing 6 derajat kebebasan, bagian tubuh atas robot (upper body) yang statis, pengujian dilakukan menggunakan motion statis mengangkat kaki kanan atau kiri, servo yang dikontrol adalah servo yang mengatur posisi roll robot, dan keseimbangan diartikan sebagai kemampuan robot untuk tidak terjatuh ke samping saat bergerak di bidang miring.

Adapun isi dari penelitian ini meliputi beberapa bagian utama. Pertama, tinjauan pustaka mengenai penggunaan sensor \emph{load cell} pada robot humanoid untuk mengukur pusat tekanan serta penggunaan kontrol \emph{PID} untuk menjaga keseimbangan robot. Kedua, desain dan implementasi sensor \emph{load cell} pada telapak kaki robot dan sistem kontrol \emph{PID} yang digunakan. Bagian hasil akan menunjukkan pembacaan sensor \emph{load cell} dan pengaruh parameter \emph{PID} terhadap keseimbangan robot. Terakhir, kesimpulan dari penelitian ini akan membahas hasil penelitian dan saran untuk penelitian selanjutnya.