\section{Kesimpulan}
\label{sec:kesimpulan}

Pada Tugas Akhir ini, dikembangkan sistem keseimbangan berbasis \emph{load cell} pada robot humanoid VI-ROSE ITS. Sistem ini menggunakan empat \emph{load cell} pada masing-masing telapak kaki robot yang diproses oleh mikrokontroler ESP32-C3 dan dikirimkan ke mikrokontroler utama melalui protokol ESP-NOW. Data tersebut digunakan untuk mendeteksi posisi pusat tekanan dan mengontrol gerakan robot agar dapat berdiri dan berjalan dengan stabil menggunakan kontrol PID. Pemilihan ESP32-C3 didasari oleh kemampuannya mengolah data dengan cepat, kompatibilitas dengan ESP32, dan ukurannya yang kecil sehingga dapat ditempatkan pada telapak kaki robot.

Hasil penelitian menunjukkan bahwa sistem keseimbangan ini efektif menjaga keseimbangan robot pada bagian \textit{roll}, terutama saat fase \textit{single phase support}, dengan memberikan nilai koreksi pada servo yang mengatur posisi \textit{roll}. Sistem ini juga berhasil mengurangi risiko robot terjatuh dan meningkatkan adaptasi terhadap permukaan miring. Pengujian kontrol PID mengungkapkan bahwa nilai optimal parameter Kp berkisar antara 0.10 hingga 0.20, yang memberikan performa terbaik dengan keberhasilan 100\% dalam menjaga keseimbangan robot. Oleh karena itu, tuning parameter Kp yang tepat sangat krusial untuk mencapai performa optimal dan menjaga stabilitas sistem robot secara keseluruhan.