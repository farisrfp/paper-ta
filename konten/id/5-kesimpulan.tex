\section{Kesimpulan}
\label{sec:kesimpulan}

Pada Tugas Akhir ini, dilakukan pengembangan sistem keseimbangan berbasis load cell pada robot humanoid dengan 29 derajat kebebasan, khususnya pada robot tari humanoid VI-ROSE ITS. Sistem tertanam pada telapak kaki, yang menggunakan mikrokontroler ESP32-C3 untuk membaca sensor load cell, berhasil dibuat dengan kesalahan pengukuran berkisar antara 0-19 gram untuk setiap load cell, dan 0-50 gram untuk total telapak kaki. Rata-rata kesalahan tercatat sebesar 14.8 gram pada kaki kanan dan 14.6 gram pada kaki kiri. Robot berhasil menyeimbangkan diri ketika mengangkat kaki kanan atau kiri dengan menerapkan kontrol PID pada lima servo yang mengatur posisi roll robot, termasuk satu servo di torso, dua servo di hip, dan dua servo di ankle. Hasil manual tuning PID menunjukkan nilai $K_p=0.1$,dan $K_d =0.005$ yang memberikan performa terbaik dengan tingkat keberhasilan 100\% dalam menjaga keseimbangan robot. Pengujian dilakukan dengan gerakan mengangkat kaki kanan maupun kaki kiri dengan kemiringan 3 derajat.

Untuk pengembangan sistem ini di masa mendatang, disarankan untuk mengganti servo dengan model yang lebih kuat, seperti MX-64T, pada motor yang mengatur posisi roll robot. Selain itu, menggunakan data posisi pusat tekanan (CoP) untuk menentukan Zero Moment Point (ZMP) dapat membantu meningkatkan akurasi keseimbangan robot.