\section{Results and Discussion}
\label{sec:resultsanddiscussion}

Testing and analysis were performed on the design and implementation previously designed. The tests include the Load Cell Sensor testing on the Robot and the Robot Balance System.

\begin{enumerate}[label=\Alph*.]

    \item Characterization Testing of Each Load Cell
    \label{subsec:results-discussion-characterization}

        \hspace*{1em} Calibration testing of the load cell sensors was conducted using five reference masses (50g, 100g, 200g, 500g, and 1000g). The smallest weight (50 grams) was used to determine the gradient coefficient, while the constant was obtained from the tare weight (zero point when the load cell has no load). The resulting data was then used to calculate the error of each load cell by comparing it with the actual mass.

        \begin{table}[h!]
            \centering
            \caption{Characterization Results of Load Cell 1}
            \begin{tabular}{|c|c|c|}
                \hline
                \textbf{Actual Weight (g)} & \textbf{Load Cell 1 Reading (g)} & \textbf{Error (g)} \\
                \hline
                50    & 50    & 0   \\
                100   & 101   & 1   \\
                200   & 202   & 2   \\
                500   & 505   & 5   \\
                1000  & 1004  & 4   \\
                \hline
            \end{tabular}
            \label{tab:Calibration_Load_Cell_1}
        \end{table}

        \begin{table}[h!]
            \centering
            \caption{Characterization Results of Load Cell 2}
            \begin{tabular}{|c|c|c|}
                \hline
                \textbf{Actual Weight (g)} & \textbf{Load Cell 2 Reading (g)} & \textbf{Error (g)} \\
                \hline
                50        & 50        & 0   \\
                100       & 100       & 0   \\
                200       & 200       & 0   \\
                500       & 500       & 0   \\
                1000      & 994       & 6   \\           
                \hline
        \end{tabular}
        \label{tab:Calibration_Load_Cell_2}
        \end{table}

        \begin{table}[h!]
            \centering
            \caption{Characterization Results of Load Cell 3}
            \begin{tabular}{|c|c|c|}
                \hline
                \textbf{Actual Weight (g)} & \textbf{Load Cell 3 Reading (g)} & \textbf{Error (g)} \\
                \hline
                50        & 50        & 0    \\    
                100       & 103       & 3    \\    
                200       & 203       & 3    \\    
                500       & 494       & 6    \\    
                1000      & 981       & 19   \\               
                \hline
        \end{tabular}
        \label{tab:Calibration_Load_Cell_3}
        \end{table}

        \begin{table}[h!]
            \centering
            \caption{Characterization Results of Load Cell 4}
            \begin{tabular}{|c|c|c|}
                \hline
                \textbf{Actual Weight (g)} & \textbf{Load Cell 4 Reading (g)} & \textbf{Error (g)} \\
                \hline
                50        & 50        & 0   \\     
                100       & 97        & 3   \\     
                200       & 204       & 4   \\     
                500       & 500       & 0   \\     
                1000      & 1003      & 3   \\                
                \hline
        \end{tabular}
        \label{tab:Calibration_Load_Cell_4}
        \end{table}
    
        
        \hspace*{1em} The characterization results for each load cell show measurement errors ranging from 0 to 19 grams. These errors are non-linear, indicating that the measurement errors are not constant across each load cell. Nevertheless, a linear equation can still be used to calculate the actual mass from the load cell readings, although it is not entirely accurate.

    \item Pressure Testing on the Footpads
    \label{subsec:results-discussion-pressure}

        \hspace*{1em} This test was conducted to obtain pressure data generated by the right and left feet when subjected to a uniform load. The pressure data generated by the right and left feet can be seen in Table \ref{tab:measurement_weight_left_foot} and Table \ref{tab:measurement_weight_right_foot}.

        \begin{table}[h!]
            \centering
            \caption{Pressure Readings for the Left Foot}
            \begin{tabular}{|c|c|c|}
                \hline
                \textbf{Actual Weight (g)} & \textbf{Reading (g)} & \textbf{Error (g)} \\
                \hline
                50    & 52    & 2   \\
                100   & 110   & 10  \\
                200   & 220   & 20  \\
                300   & 304   & 4   \\
                500   & 512   & 12  \\
                700   & 701   & 1   \\
                1000  & 1050  & 50  \\
                1300  & 1325  & 25  \\
                1500  & 1512  & 12  \\
                1800  & 1788  & 12  \\
                \hline
                \textbf{Average Error (g)} & \multicolumn{2}{c|}{\textbf{14.8}} \\
                \hline
            \end{tabular}
            \label{tab:measurement_weight_left_foot}
        \end{table}

        \begin{table}[h!]
            \centering
            \caption{Pressure Readings for the Right Foot}
            \begin{tabular}{|c|c|c|}
                \hline
                \textbf{Actual Weight (g)} & \textbf{Reading (g)} & \textbf{Error (g)} \\
                \hline
                50    & 46    & 4    \\
                100   & 98    & 2    \\
                200   & 215   & 15   \\
                300   & 325   & 25   \\
                500   & 505   & 5    \\
                700   & 722   & 22   \\
                1000  & 1025  & 25   \\
                1300  & 1347  & 47   \\
                1500  & 1500  & 0    \\
                1800  & 1819  & 19   \\
                \hline
                \textbf{Average Error (g)} & \multicolumn{2}{c|}{\textbf{16.4}} \\
                \hline
            \end{tabular}
            \label{tab:measurement_weight_right_foot}
        \end{table}

        \begin{figure}[h]
            \centering
            \includegraphics[width=0.4\textwidth]{gambar/chart_tekanan_kaki.png}
            \caption{Graph of the Relationship Between Actual Weight and Load Cell Readings for the Left and Right Feet}
            \label{fig:foot_pressure}
        \end{figure}

        \hspace*{1em} Figure \ref{fig:foot_pressure} shows that the pressure graphs produced by the right and left feet have similar reading values. The pressure measurement results for the left and right feet show that the load cell readings on both feet have errors ranging from 0 to 50 grams. These errors are due to differences in characteristics between the load cells on the left and right feet. 

\newpage

    \item Center of Pressure Testing on the Robot
    \label{subsec:hasil-pembahasan-pusat-tekanan}
        
        \hspace*{1em} This test was conducted when the robot performed motions and collected input data from the load cell sensor at 50 ms intervals. The aim of this test was to obtain center of pressure data while the robot was performing motions. The test was conducted as shown in Figure \ref{fig:pusat_tekanan_robot} by collecting 3 center of pressure data points while the robot performed in-place walking motions, namely when the robot was in the double support position, lifting the right foot, and lifting the left foot.
        
        \begin{figure}[h]
            \centering
            \includegraphics[width=0.4\textwidth]{gambar/hasil/jalan_ditempat.png}
            \caption{Robot Posture When Walking in Place \\ (a) Double Phase Support \\ (b) Lifting Right Foot \\ (c) Lifting Left Foot}
            \label{fig:pusat_tekanan_robot}
        \end{figure}
        
        \hspace*{1em} To display the center of pressure data generated by the robot, a visualization of the robot's foot was used. The center of pressure position on the X and Y axes is displayed as coordinate points with a scale of 2:1. This visualization helps in understanding the center of pressure position on the robot when performing in-place walking motions. Data was collected based on 3 center of pressure data points generated in the previous scenario.
        
        \hspace*{1em} In Figure \ref{fig:pusat_tekanan_robot_double}, the center of pressure on the X-axis is 0.09 and on the Y-axis is 0.01. In Figure \ref{fig:pusat_tekanan_robot_angkat_kiri}, the center of pressure on the X-axis is -1.30 and on the Y-axis is 0.16. Meanwhile, in Figure \ref{fig:pusat_tekanan_robot_angkat_kanan}, the center of pressure on the X-axis is 1.44 and on the Y-axis is 0.03. From this visualization, the center of pressure on the X-axis changes significantly when the robot lifts the left and right foot, while on the Y-axis, the center of pressure coordinates remain in the middle of the foot.
        
        \begin{figure}[h]
            \centering
            \includegraphics[width=0.3\textwidth]{gambar/hasil/cop_double_phase.png}
            \caption{Robot COP When in Double Phase Support}
            \label{fig:pusat_tekanan_robot_double}
        \end{figure}

\newpage
        
        \begin{figure}[h]
            \centering
            \includegraphics[width=0.3\textwidth]{gambar/hasil/cop_angkat_kiri.png}
            \caption{Robot COP When Lifting Left Foot}
            \label{fig:pusat_tekanan_robot_angkat_kiri}
        \end{figure}
        
        \begin{figure}[h]
            \centering
            \includegraphics[width=0.3\textwidth]{gambar/hasil/cop_angkat_kanan.png}
            \caption{Robot COP When Lifting Right Foot}
            \label{fig:pusat_tekanan_robot_angkat_kanan}
        \end{figure}

    
    \item PID Control System Testing
    \label{subsec:hasil-pembahasan-pid}
        
        \hspace*{1em} In this test, the robot's balance system was tested using PID control. The results to be analyzed are the system response when using PID control and without PID control. The test was conducted by lifting the right or left foot with a 3-degree tilt.

        \hspace*{1em} Figure \ref{fig:robot_not_fall} shows the test results with the PID controller. Meanwhile, Figure \ref{fig:robot_fall} shows the results without the PID controller. In Figure \ref{fig:robot_not_fall}, if the center of pressure value on the X-axis exceeds the maximum limit within a certain time, the robot will fall. This indicates that the error value produced by the PID controller cannot be well calculated because the input value received by the PID controller does not match the expected value.

        \begin{figure}[h]
            \centering
            \includegraphics[width=0.4\textwidth]{gambar/hasil/hasil_pid.png}
            \caption{System Response Graph With PID Control (Not Fall)}
            \label{fig:robot_not_fall}
        \end{figure}

\newpage
        
        \begin{figure}[h]
            \centering
            \includegraphics[width=0.4\textwidth]{gambar/hasil/hasil_no_pid.png}
            \caption{System Response Graph Without PID Control (Fall)}
            \label{fig:robot_fall}
        \end{figure}

        
    \item Influence of PID Parameters Testing
    \label{subsec:hasil-pembahasan-parameter-pid}
        
        \hspace*{1em} In this test, the influence of each PID parameter was analyzed. The results to be analyzed are the influence of PID parameters on the system response and the Root Mean Square (RMS) error values produced by the system. The test was conducted similarly to the previous test, by lifting the right or left foot with a 3-degree tilt.
        
        \begin{table}[h]
            \centering
            \caption{Effect of Parameter $K_p$ on Right Foot Lifting with 3-Degree Tilt}
            \begin{tabular}{|c|c|c|c|c|}
                \hline
                \textbf{PID} & \textbf{Fall} & \textbf{Not Fall} & \textbf{Success} & RMS Error \\
                \hline
                $K_p = 0.00$ & 6 & 0 & 0   \% & 0.7598 \\
                $K_p = 0.05$ & 6 & 0 & 0   \% & 0.7690 \\
                $K_p = 0.10$ & 0 & 6 & 100 \% & 0.7779 \\
                $K_p = 0.15$ & 1 & 5 & 83  \% & 0.8145 \\
                $K_p = 0.20$ & 1 & 5 & 83  \% & 0.8870 \\
                $K_p = 0.25$ & 0 & 6 & 100 \% & 0.8801 \\
                \hline
            \end{tabular}
            \label{tab:testing_p}
        \end{table}

    \hspace*{1em} The results shown in Table \ref{tab:testing_p} indicate that the optimal $K_p$ parameter value for maintaining robot balance ranges from 0.10 to 0.20. At $K_p = 0.00$ and $K_p = 0.05$, all tests failed, and the robot consistently fell. The $K_p = 0.10$ value shows the best performance with a 100\% success rate in maintaining balance during right and left foot lifting at a 3-degree tilt. Higher $K_p$ values, such as $K_p = 0.25$, show a decline in performance. The Root Mean Square (RMS) results indicate the smallest error at $K_p = 0.10$ with a value of 0.7779, highlighting the importance of setting an appropriate $K_p$ value for robot stability.

    \begin{table}[h]
        \centering
        \caption{Effect of Parameter $K_i$ on Right Foot Lifting with 3-Degree Tilt}
        \begin{tabular}{|c|c|c|c|c|}
            \hline
            \textbf{PID} & \textbf{Fall} & \textbf{Not Fall} & \textbf{Success} & RMS Error \\
            \hline
            $K_p = 0.1, K_i = 0.01$ & 2 & 4 & 66 \%  & 0.9701\\
            $K_p = 0.1, K_i = 0.02$ & 2 & 4 & 66 \%  & 0.8950\\
            $K_p = 0.1, K_i = 0.04$ & 1 & 5 & 83 \%  & 0.9345\\
            $K_p = 0.1, K_i = 0.10$ & 0 & 6 & 100 \% & 0.8471\\
            $K_p = 0.1, K_i = 0.20$ & 0 & 6 & 100 \% & 0.8980\\           
            \hline
        \end{tabular}
        \label{tab:testing_pi}
    \end{table}

    \hspace*{1em} The results shown in Table \ref{tab:testing_pi} indicate that the optimal $K_i$ parameter value for maintaining robot balance ranges from 0.10 to 0.20. At $K_i = 0.01$ and $K_i = 0.02$, the robot failed to maintain balance with a success rate of 66\% to 83\%. Conversely, at $K_i = 0.10$ and $K_i = 0.20$, the robot successfully maintained balance with a 100\% success rate during right and left foot lifting at a 3-degree tilt. The Root Mean Square (RMS) results show the lowest value at $K_i = 0.10$ with 0.8471, while the highest value at $K_i = 0.01$ is 0.9701. The influence of $K_i$ on performance is not very significant, and in some cases, higher $K_i$ values reduce system performance, indicating that the $K_i$ parameter in the PID control for this system is not very critical.

    \begin{table}[h]
        \centering
        \caption{Effect of Parameter $K_d$ on Right Foot Lifting with 3-Degree Tilt}
        \begin{tabular}{|c|c|c|c|c|}
            \hline
            \textbf{PID} & \textbf{Fall} & \textbf{Not Fall} & \textbf{Success} & RMS Error \\
            \hline
            $K_p = 0.1, K_d = 0.005$ & 0 & 6 & 100 \% & 0.7143 \\
            $K_p = 0.1, K_d = 0.010$ & 0 & 6 & 100 \% & 0.7077 \\
            $K_p = 0.1, K_d = 0.020$ & 2 & 4 & 66  \% & 0.7262 \\
            $K_p = 0.1, K_d = 0.050$ & 5 & 1 & 16  \% & 0.7344 \\
            $K_p = 0.1, K_d = 0.100$ & 6 & 0 & 0   \% & 0.9231 \\          
            \hline
        \end{tabular}
        \label{tab:testing_pd}
    \end{table}

    \hspace*{1em} The results shown in Table \ref{tab:testing_pd} indicate that the optimal $K_d$ parameter value for maintaining robot balance ranges from 0.005 to 0.020. At $K_d = 0.005$ and $K_d = 0.010$, the robot successfully maintained balance with a 100\% success rate. However, at $K_d = 0.020$, the success rate decreased to 66\%, and at $K_d = 0.050$, the success rate was only 16\%. At $K_d = 0.100$, the robot consistently fell. The Root Mean Square (RMS) results show the lowest value at $K_d = 0.010$ with 0.7077, while the highest value at $K_d = 0.100$ is 0.9231. The impact of $K_d$ on system performance is quite significant, with higher $K_d$ values reducing system performance.


\end{enumerate}
