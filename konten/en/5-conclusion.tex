\section{Conclusion}
\label{sec:conclusion}

In this final project, a balance system based on load cells was developed for the VI-ROSE ITS humanoid robot. This system uses four load cells on each foot sole of the robot, processed by an ESP32-C3 microcontroller and transmitted to the main microcontroller via the ESP-NOW protocol. This data is used to detect the center of pressure position and control the robot's movements to stand and walk stably using PID control. The choice of ESP32-C3 is based on its ability to process data quickly, its compatibility with ESP32, and its small size, allowing it to be placed on the robot's foot sole.

The research results show that this balance system is effective in maintaining the robot's balance in the roll aspect, especially during the single-phase support phase, by providing corrective values to the servo that controls the roll position. This system also successfully reduces the risk of the robot falling and increases adaptation to inclined surfaces. PID control testing revealed that the optimal value of the Kp parameter ranges from 0.10 to 0.20, providing the best performance with a 100\% success rate in maintaining the robot's balance. Therefore, proper tuning of the Kp parameter is crucial to achieve optimal performance and maintain the overall stability of the robot system.
