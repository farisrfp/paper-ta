\section{Conclusion}
\label{sec:conclusion}

In this study, a balance system based on load cells was developed for a humanoid robot with 29 degrees of freedom, specifically for the VI-ROSE ITS humanoid dance robot. The system, embedded in the soles of the feet and using the ESP32-C3 microcontroller to read the load cell sensors, was successfully created with measurement errors ranging from 0-19 grams per load cell and 0-50 grams for the total foot. The average errors recorded were 14.8 grams for the right foot and 14.6 grams for the left foot. The robot successfully balanced itself while lifting either the right or left foot by applying PID control to five servos that adjust the robot's roll position, including one servo in the torso, two servos in the hip, and two servos in the ankle. Manual PID tuning results showed that $K_p = 0.1$ and $K_d = 0.005$ provided the best performance, achieving a 100\% success rate in maintaining robot balance. Testing was performed with foot lifting movements at a 3-degree tilt.

For future development of this system, it is recommended to replace the servos with more robust models, such as MX-64T, for the motors that control the robot's roll position. Additionally, using Center of Pressure (CoP) data to determine the Zero Moment Point (ZMP) could help improve the robot's balance accuracy.