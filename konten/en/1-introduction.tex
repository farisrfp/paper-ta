% Change the following title and label as desired.
\section{Introduction}
\label{sec:introduction}

In recent years, rapid developments in the field of robotics have transformed the interaction between humans and robots, especially in the context of humanoid robots\cite{chiang2020posture}. One of the key aspects in the design of humanoid robots is maintaining balance and stability, particularly when the robot performs activities involving leg movements with support on one leg. This becomes crucial in robot dance competitions, where robots must maintain balance and walk on flat surfaces with varying heights without falling.

Previous research has shown various approaches to achieve balance in humanoid robots. For example, Farah Risha (2023)\cite{farah} used IMU sensors to obtain pitch tilt angle data, which is then processed by a microcontroller to adjust the angle on the robot's servo. On the other hand, Arifin (2017)\cite{arifin2017implementasi} studied the use of pressure sensors on the soles of the feet as part of the balance control system. The main challenge in this research was the inefficiency of data transmission through cables, which is susceptible to disturbances from leg movements.

The main issue faced is that the robot tends to fall sideways when disturbances occur while standing on one leg. Therefore, it is necessary to improve the system's response to balance changes. In this study, load cell sensors will be used to control roll movements due to their more accurate position in detecting pressure changes on the robot's soles. Pressure sensors offer a more accurate solution for measuring changes in the robot's center of pressure, allowing the system to respond quickly. The implementation of wireless technology based on 2.4GHz radio signals is proposed to improve the efficiency of data transmission from the pressure sensors on the soles.

This research aims to develop a balance system in humanoid robots that utilizes pressure sensors from load cells, focusing on improving the robot's stability when standing on one leg. The problem constraints set include the static nature of the robot's upper body, the number of degrees of freedom in the legs, the use of the ESP32 microcontroller, and the definition of balance as the robot's ability to avoid falling while moving on uneven surfaces.
