% Change the following title and label as desired.
\section{Introduction}
\label{sec:introduction}

In recent years, rapid advancements in robotics have increased interactions between humans and robots, especially in the context of humanoid robots. Maintaining balance and stability is a key aspect in the design of humanoid robots\cite{chiang2020posture}. This is particularly important in traditional dance robot competitions, where robots are not allowed to fall to cross the arena and must be able to perform dances with complex movements. Humans have the ability to balance quickly and efficiently, even when walking on uneven surfaces.

Research by Ilyasaa (2023)\cite{rotama}, discusses the implementation of load cell sensors on humanoid robots with 24 degrees of freedom to control balance while walking. This sensor detects the pressure on the robot's feet and is used as a PID control input to regulate the robot's step pitch parameters. The results show that the robot can stand at a tilt angle of 0°-25° and control the pitch to reduce the likelihood of falling, but at 30°, the robot experiences difficulty and tends to fall. The advantage of this research is the effective use of PID control, but it lacks an analysis of the impact of tilt angles on the robot's control while walking.

Wulandari (2023)\cite{lanangjagad} discusses the use of Load Cell sensors, Kalman Filter, and PID Controller for the balance of a dancing humanoid robot, which must be stable in various competition field conditions (KRSTI). The control system uses MPU6050 and Load Cell sensors to detect the Center of Pressure (CoP) and maintain balance. The results show the robot successfully maintains balance with an 87.5\% success rate when standing and 89\% when dancing. The advantage of this research is the effective combination of sensors and algorithms, but the testing was only conducted in the double support phase, without considering movements in the single support phase, and the hardware design still uses cables, restricting the robot's foot movements.

The robot being studied still occasionally falls when dancing, especially when walking or lifting one foot. In humanoid dance robots commonly used for the Dance Robot Art Contest (KRSTI), the addition of accessories to the costume further challenges the system's ability to maintain balance. To overcome this, it is necessary to improve the system's response to balance changes. Load cell sensors can provide information about the robot's center of pressure. These sensors can detect detailed pressure changes, allowing the system to measure changes in the balance center more precisely and respond quickly. Additionally, to improve data transmission efficiency from the load cell sensors located on the soles, the implementation of wireless technology using 2.4GHz radio signals is proposed.

This research aims to develop an embedded system on the soles of a humanoid dance robot with 29 degrees of freedom capable of reading the pressure center using load cell sensors and applying a control system to control several servos to balance the robot when moving on inclined surfaces. The research limitations include the use of the ESP32 microcontroller, the number of degrees of freedom of the robot's legs with 6 degrees of freedom each, the static upper body of the robot, testing conducted using static motion lifting the right or left foot, the servos controlled are those that regulate the robot's roll position, and balance is defined as the robot's ability not to fall sideways when moving on inclined surfaces.

The contents of this research include several main parts. First, a literature review on the use of load cell sensors in humanoid robots to measure the pressure center and the use of PID control to maintain the robot's balance. Second, the design and implementation of load cell sensors on the robot's soles and the PID control system used. The results section will show the load cell sensor readings and the effect of PID parameters on the robot's balance. Finally, the conclusion of this research will discuss the research findings and suggestions for future research.